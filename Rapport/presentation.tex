\documentclass[11pt]{beamer}
 
\usepackage[utf8]{inputenc} 
\usepackage[T1]{fontenc}
\usepackage{times}
\usepackage{graphicx}
\usepackage[french]{babel}
 
\usetheme{Madrid}


\title{Sujet: Soko'banzai}
\subtitle{Projet tuteuré}
\author
{
Guillaume \textsc{Drouart}\\
Marwan \textsc{Lakradi}
}
\institute
{
Année universitaire 2015-2016\\
Diplôme préparé: L3 informatique\\[0.5cm]
\includegraphics[height=10mm]{logo-ucbn.png}
}
%\date{}

\begin{document}
\begin{frame}
\titlepage
\end{frame}

\AtBeginSection[]{
\begin{frame}{Sommaire}
  \begin{columns}[t]
  \begin{column}{5cm}
  \tableofcontents[sections={1-3},currentsection, hideothersubsections]
  \end{column}
  \begin{column}{5cm}
  \tableofcontents[sections={4-6},currentsection,hideothersubsections]
  \end{column}
  \end{columns}
\end{frame}
}

\section{Introduction}
\begin{frame}
\frametitle{Introduction}
\begin{block}{Quel est notre projet ?}
\begin{itemize}
\item Un jeu de Sokoban
\item Un travail de groupe
\item Une intelligence artificielle
\end{itemize}
\end{block}
\end{frame}

\section{Objectifs et Cahier des charges}
\subsection{Objectifs}
\begin{frame}
\frametitle{Objectifs}
\begin{itemize}
\setbeamertemplate{itemize item}[triangle]
\item Permettre à un utilisateur de jouer à notre jeu
\item L'utilisateur peut demander à une intelligence artificielle de résoudre un niveau
\item Penser la conception du programme et établir les choix techniques à utiliser
\item Avoir une application fonctionnelle, évolutive et modulaire 
\end{itemize}
\end{frame}

\subsection{Cahier des charges}
\begin{frame}
\frametitle{Cahier des charges}
\begin{columns}[t] 
\begin{column}{6cm}
\begin{itemize}
\setbeamertemplate{itemize item}[triangle]
\item Importer,déplacer et repositionner les cartes
\item Modifier la rotation des cartes 
\item Gérer l’affichage des cartes 
\item Gérer une main 
\end{itemize}
\end{column}
\begin{column}{6cm}
\begin{itemize}
\setbeamertemplate{itemize item}[triangle]
\item Construire des piles de cartes
\item Ajouter et retirer une carte de la pile
\item Mélanger les cartes d’une pile
\item Déplacer la pile
\end{itemize}
\end{column}
\end{columns}
\end{frame}

\section{Langages et Bibliothèques}

\subsection{Langages}
\begin{frame}
\frametitle{Langages}
\begin{block}{Langages utilisés}
\begin{itemize}
\item HTML5 et CSS3
\item Le Javascript
\end{itemize}
\end{block}
\begin{figure}
%\includegraphics[height=20mm]{htmlJsCss.jpg}
\end{figure}
\end{frame}

\subsection{Bibliothèques}
\begin{frame}
\frametitle{Bibliothèques}
\begin{block}{Bibliothèques}
\begin{itemize}
\item JQuery
\end{itemize}
\end{block}
\begin{exampleblock}{Extension et Plugins}
\begin{itemize}
\item JQueryUI
\item Drag and Drop
\item Rotate
\end{itemize}
\end{exampleblock}
\begin{figure}
%\includegraphics[height=8mm]{JQuery_logo.png}
%\includegraphics[height=8mm]{JQuery_UI_Logo.png}
\end{figure}
\end{frame}


\section{Réalisation, Architecture de l'application}

\subsection{Interface}
\begin{frame}
\frametitle{Interface}
\begin{figure}
%\includegraphics[height=65mm]{int.jpg}
\end{figure}
\end{frame}

\subsection{Réalisation}
\begin{frame}
\frametitle{Réalisation}
\begin{itemize}
\item Utilisation d'un logiciel de versioning : SVN
\item Répartition des tâches lors de la séance de TPA en fonction des affinités de chacun
\item Réalisation de tests unitaires
\end{itemize}
\end{frame}

\subsection{Architecture de l'application}
\begin{frame}
\frametitle{Architecture de l'application}
%\includegraphics[height=50mm]{lang.jpg}
\end{frame}

\subsection{Démonstration}
\begin{frame}
\frametitle{Démonstration}
\begin{figure}
%\includegraphics[height=65mm]{site.png}
\end{figure}
\end{frame}

\section{Problèmes rencontrés}
\begin{frame}
\frametitle{Problèmes rencontrés}
\begin{itemize}
\item Quelques problèmes de communication 
\item Problèmes avec l'utilisation de la POO, concept abandonné au final dans le projet
\item Problèmes avec le déplacement de paquets de cartes 
\end{itemize}
\end{frame}


\section{Conclusion}
\begin{frame}
\frametitle{Conclusion}
\begin{columns}[b]
\begin{column}{5.8cm}
\begin{block}{Résultats}
\begin{itemize}
\item Objectifs atteints
\item Interface complète
\item Les fonctionnalités sont disponibles 
\item Utilisation de bibliothèques
\item Module d'importation des cartes fonctionnel
\end{itemize}
\end{block}
\end{column}
\begin{column}{5.8cm}
\begin{block}{Améliorations envisagées}
\begin{itemize}
\item Zoomer sur les cartes
\item Importer un fichier zip de cartes
\item Ouvrir un jeu de carte par défaut
\item Sauvegarder une partie en cours
\end{itemize}
\end{block}
\end{column}
\end{columns}
\end{frame}

\end{document}